\documentclass[11pt,letterpaper]{article}

\pagestyle{plain}                               
\usepackage{hyperref}
\hypersetup{
    colorlinks=true,
    linkcolor=black,
    filecolor=magenta,      
    urlcolor=blue,
}             

\setlength{\textwidth}{6.5in}     
\setlength{\oddsidemargin}{0in}   
\setlength{\evensidemargin}{0in}  
\setlength{\textheight}{9.0in}    
\setlength{\topmargin}{0in}       
\setlength{\headheight}{0in}      
\setlength{\headsep}{0in}         
\setlength{\footskip}{.5in}       
\setlength{\parskip}{5pt}
\setlength{\parindent}{0pt}

\usepackage{graphicx}
\usepackage{amsmath,amsbsy,amssymb,amsfonts,bm,siunitx,float}
%PDE COMMANDS% 

%1ST ORDER PARTIAL
\newcommand{\fdel}[2]{\dfrac{\partial{#1}}{\partial {#2}}} %\fdel{function}{var}

%ONE VARIABLE 2ND ORDER PARTIAL
\newcommand{\sdel}[2]{\dfrac{\partial^2{#1}}{\partial{#2}^2}} %\sdel{function}{var}

%ORDERED 2ND ORDER PARTIAL
\newcommand{\osdel}[3]{\dfrac{\partial^2{#1}}{\partial{#2}\partial{#3}}} %\osdel{function}{var1}{var2}

%FITTED (), [], {}, ||%
\newcommand{\fpar}[1]{\left({#1}\right)} %()
\newcommand{\fbrac}[1]{\left[{#1}\right]} %[]
\newcommand{\fset}[1]{\left\{{#1}\right\}} %{}
\newcommand{\fabs}[1]{\left|{#1}\right|} % ||
\newcommand{\fang}[1]{\left\langle{#1}\right\rangle}

%SET COMMANDS%
\newcommand{\N}{\mathbb N} %Natural Numbers
\newcommand{\R}{\mathbb R} %Real Numbers
\newcommand{\Q}{\mathbb Q} %Rational Numbers
\newcommand{\I}{\mathbb{Q}^c}
\newcommand{\Z}{\mathbb Z} %Integers
\newcommand{\sP}{{\mathscr P}} %Power Set

%vector shit%
\newcommand{\unit}[1]{\hat{\mathbf{#1}}} %Bold letter with hat overtop
\newcommand{\del}{\pmb{\nabla}} %Bold del
\newcommand{\curl}{\pmb\del\times}
%QUANTUM COMMANDS%
%Operators
%Identity Operator


%GRAPHICS PATH TO FOLDER NAMED Figures%
\graphicspath{ {./Figures/} }

\title{Analytical and Numerical Analysis of Fluid Flow In A Clyindrical Container Using The Navier-Stokes Equations}
\author{Nick Navas}
\begin{document}

\maketitle

\section{Introduction}
Here we will consider the flow of incompressible fluids with kinematic viscosity $\nu$ in a no-slip, circular cylinder of height $H$, radius $R$ driven by the bottom endwall with a constant angular velocity $\Omega$. This allows us to define on the length scale of $R$ and time scale $1/\Omega$, we have the Reynolds number, $Re = \Omega R^2 / \nu$. In the field of fluid dynamics the non-dimensional Navier-Stokes equations describe the flow of incompressible fluids and are given as
\begin{equation}
    \fdel{\mathbf{u}}{t} + \mathbf{u} \cdot \del\mathbf{u} = -\del p + \dfrac{1}{Re}\nabla^2\mathbf{u},
\end{equation}
\begin{equation}
    \del \cdot \mathbf{u} = 0,
\end{equation}
for the flow velocity of the fluid $\mathbf{u} = \left\langle \fdel{q_1}{t},\fdel{q_2}{t},\fdel{q_3}{t} \right\rangle$ (in generalized coordinates). Here equation (1) represents the conservation of momentum of our system and equation (2) states that our fluid is incompressible. Our goal here is to get this into a form where we can apply methods for solving partial differential equations. Since our system has circular cylindrical symmetry we can map $\mathbf{u}$ to circular cylindrical coordinates as the vector $\langle{\dot{r},\dot{\theta},\dot{z}\rangle} = \langle{u,v,w\rangle}$. With this choice of coordinate system it is useful to note the gradient, Laplacian, divergence, and curl in cylidrical coordinates,\footnote{John Taylor \emph{Classical Mechanics} back cover}
\begin{align}
    \del  &= \unit{r}\fdel{}{r} + \pmb{\hat{\theta}}\frac{1}{r} \fdel{}{\theta}+\unit{z}\fdel{}{z}\\
    \nabla^2 &= \sdel{}{r}+\frac{1}{r}\fdel{}{r}+\frac{1}{r^2}\sdel{}{\theta}+\sdel{}{z}\\
    \del\cdot\mathbf{A} &= \frac{1}{r}\fdel{}{r}(r\ A_r)+\frac{1}{r}\fdel{A_\theta}{\theta}+\fdel{A_z}{z}\\
    \del \times \mathbf{A} &= \left\langle\fpar{\frac{1}{r}\fdel{A_z}{\theta} - \fdel{A_\theta}{z}},\fpar{\fdel{A_r}{z} - \fdel{A_z}{r}},\frac{1}{r}\fpar{\fdel{}{r}\fpar{r\ A_\theta} - \fdel{A_r}{\theta}}\right\rangle.
\end{align} 
Next we are going to want to define what $\nabla^2\mathbf{u}$ and $\mathbf{u}\cdot\del\mathbf{u}$ are in cylindrical coordinates. We can write
\begin{equation}
    \nabla^2\mathbf{u} = \del(\del\cdot\mathbf{u})- \del\times(\del\times\mathbf{u}).\footnote{ASU Physics Department \emph{Tutoral: Vector Calculus}, equation 29.}
\end{equation} 
When applying equations (3), (5), and (6) to $\del(\del\cdot\mathbf{u})- \del\times(\del\times\mathbf{u})$ we get (in component form),
\begin{align}
    \del(\del\cdot\mathbf{u})- \del\times(\del\times\mathbf{u}) &= 
    \left\langle\fpar{\sdel{u}{r}+\dfrac{1}{r}\fdel{u}{r}+\dfrac{1}{r^2}\sdel{u}{\theta} -\dfrac{2}{r^2}\fdel{u}{\theta} + \sdel{u}{z}-\dfrac{u}{r^2}},\right.\nonumber  \\
    &\ \ \ \ \ \ \fpar{\sdel{v}{r} + \dfrac{1}{r^2}\sdel{v}{\theta} +\sdel{v}{z} + \dfrac{1}{r}\fdel{v}{r}+\dfrac{2}{r^2}\fdel{v}{\theta} -\dfrac{v}{r^2}},\nonumber\\
    &\ \ \ \ \ \ \left.\fpar{\sdel{w}{r} +\dfrac{1}{r^2}\sdel{w}{\theta}+\sdel{w}{z}+\dfrac{1}{r}\fdel{w}{z} }\right\rangle.
\end{align}
We know that $\mathbf{u}\cdot\del\mathbf{u}$ In cylindrical coordinates\footnote{Weisstein, Eric W. "Convective Operator." \href{https://mathworld.wolfram.com/ConvectiveOperator.html}{From MathWorld--A Wolfram Web Resource.}} is, 
\begin{align}
    \mathbf{u}\cdot\del\mathbf{u} &= \left\langle \fpar{u\fdel{u}{r} +\dfrac{v}{r}\fdel{u}{\theta} +w\fdel{u}{z}-\dfrac{v^2}{r}} \right.\nonumber\\
    & \ \ \ \ \ \ \ \fpar{u\fdel{v}{r} +\dfrac{v}{r}\fdel{v}{\theta} +w\fdel{v}{z}+\dfrac{v^2}{r}},\nonumber\\
    & \ \ \ \ \ \ \ \left. \fpar{u\fdel{w}{r}+\frac{v}{r}\fdel{w}{\theta}+w\fdel{w}{z}} \right\rangle.
\end{align}
For the following we will assume that the Reynolds number is sufficiently small $(Re\lesssim 10^3)$ which implies that the flow is steady and axisymmetric (i.e. $\partial/\partial t = 0$ and $\partial/\partial\theta = 0$ respectively).
Using the results from equations (3), (8), and (9) into equation (1) for low Reynolds numbers we will have the following components of the axisymmetric version of equation (1) with an aspect ration $ H/R = \Gamma  \sim \mathcal{O}(1)$,
\begin{align}
    \fdel{u}{t}+u\fdel{u}{r} +w\fdel{u}{z} - \dfrac{v^2}{r} &= -\fdel{p}{r}+\dfrac{1}{Re}\fbrac{{\sdel{u}{r}+\dfrac{1}{r}\fdel{u}{r}+ \sdel{u}{z}-\dfrac{u}{r^2}}}\\
    \fdel{v}{t} +u\fpar{\fdel{v}{r}+\dfrac{v}{r}} +w\fdel{v}{z} &= \dfrac{1}{Re}\fbrac{{\sdel{v}{r}+\dfrac{1}{r}\fdel{v}{r}+ \sdel{v}{z}-\dfrac{v}{r^2}}}\\
    \fdel{w}{t} +u\fdel{w}{r}+w\fdel{w}{z} &= -\fdel{p}{r}+\dfrac{1}{Re}\fbrac{{\sdel{w}{r}+\dfrac{1}{r}\fdel{w}{r}+ \sdel{w}{z}}}.
\end{align}
We can also get the axisymmetric version of equation (2) by applying the results from equation (5) to get,
\begin{equation}
    \dfrac{1}{r}\fdel{}{r}\fpar{ru} + \fdel{w}{z} = 0.
\end{equation}
Due to the axisymmetry, it is also convenient to define the
streamfunction $\psi$ such that,
\begin{equation*}
    \fang{u,v,w} = \fang{-\dfrac{1}{r}\fdel{\psi}{z},v,\dfrac{1}{r}\fdel{\psi}{r}}.
\end{equation*}
With this we can show that this satisfies the incompressibiblity condition equation (13)  
\begin{align*}
    \dfrac{1}{r}\fdel{}{r}\fpar{ru} + \fdel{w}{z} &= \dfrac{1}{r}\fdel{}{r}\fpar{-r\dfrac{1}{r}\fdel{\psi}{z}} + \dfrac{1}{r}\fdel{}{z}\fpar{\fdel{\psi}{r}}\\
    &=-\dfrac{1}{r}\osdel{\psi}{r}{z} + \dfrac{1}{r}\osdel{\psi}{z}{r}\\
    &=0.
\end{align*}
Another useful quantity to define is the vorticity of the fluid. This can be defined as $\curl\mathbf{u}$. Applying the result of equation (6) we have the vorticity being,
\begin{equation*}
     \curl\mathbf{u} = \fang{-\fdel{v}{z},\fdel{u}{z}-\fdel{w}{r},\dfrac{1}{r}\fdel{}{r}\fpar{rv}}.
\end{equation*}
For notation sake we take 
\begin{align}
    \eta &=\fdel{u}{z}-\fdel{w}{r} \nonumber\\
    &=-\dfrac{1}{r}\sdel{\psi}{z} -\dfrac{1}{r}\sdel{\psi}{r}+\dfrac{1}{r^2}\fdel{\psi}{r}\nonumber,
\end{align}
which we get by substituting the streamfunction into $\eta$. This leads to the implication that, 
\begin{equation}
   \sdel{\psi}{z} +\sdel{\psi}{r}-\fdel{\psi}{r} = -r\eta.
\end{equation}
Looking at equations (10) and (12) we can eliminate the constraint on pressure $p$ by doing,
\begin{equation*}
    \fdel{(10)}{z}-\fdel{(12)}{r},
\end{equation*} 
resulting in the left hand side of the equation being
\begin{align} 
    \fdel{}{t}\fbrac{\fdel{u}{z}-\fdel{w}{r}} + u\fdel{}{r}\fpar{\fdel{u}{z}-\fdel{w}{z}} +w\fdel{}{z}\fpar{\fdel{u}{z} - \fdel{w}{r}} + \fdel{u}{r}\fpar{\fdel{u}{z} - \fdel{w}{r}} - \dfrac{2v}{r}\fdel{v}{z},\nonumber
\end{align}
which when applying the streamfunction becomes 
\begin{align*}
    \fdel{\eta}{t} -\dfrac{1}{r}\fdel{\psi}{z}\fdel{\eta}{r}+\dfrac{1}{r}\fdel{\psi}{r}\fdel{\eta}{z} +\dfrac{\eta}{r^2}\fdel{\psi}{z}-\dfrac{2v}{r}\fdel{v}{z}.
\end{align*}
Now examining the right hand side it becomes
\begin{align}
    \dfrac{1}{Re}\left[\sdel{}{r}\fpar{\fdel{u}{z} - \fdel{w}{r}}+ \dfrac{1}{r}\fdel{}{r}\fpar{\fdel{u}{z}-\fdel{w}{r}}\right.-\left.\dfrac{1}{r^2}\fpar{\fdel{u}{z}-\fdel{w}{r}} + \sdel{}{z}\fpar{\fdel{u}{z}-\fdel{w}{r} }\right],\nonumber
\end{align}
which after applying the streamfunction gives
\begin{align*}
    \dfrac{1}{Re} \fpar{\sdel{\eta}{r} + \dfrac{1}{r}\fdel{\eta}{r}-\dfrac{\eta}{r^2}+\sdel{\eta}{z}},
\end{align*}
resulting in
\begin{equation}
    \fdel{\eta}{t} -\dfrac{1}{r}\fdel{\psi}{z}\fdel{\eta}{r}+\dfrac{1}{r}\fdel{\psi}{r}\fdel{\eta}{z} +\dfrac{\eta}{r^2}\fdel{\psi}{z}-\dfrac{2v}{r}\fdel{v}{z}=\dfrac{1}{Re} \fpar{\sdel{\eta}{r} + \dfrac{1}{r}\fdel{\eta}{r}-\dfrac{\eta}{r^2}+\sdel{\eta}{z}}.
\end{equation}.
We also can write equation (11) in terms of the streamfunction which yields 
\begin{align}
    \fdel{v}{t} -\dfrac{1}{r}\fdel{\psi}{z}\fpar{\fdel{v}{r}+\dfrac{v}{r}} +\dfrac{1}{r}\fdel{\psi}{r}\fdel{v}{z} &=   \dfrac{1}{Re}\fbrac{{\sdel{v}{r}+\dfrac{1}{r}\fdel{v}{r}+ \sdel{v}{z}-\dfrac{v}{r^2}}}\nonumber\\
    Re\fbrac{\fdel{v}{t} -\dfrac{1}{r}\fdel{\psi}{z}\fpar{\fdel{v}{r}+\dfrac{v}{r}} +\dfrac{1}{r}\fdel{\psi}{r}\fdel{v}{z}} &=   {{\sdel{v}{r}+\dfrac{1}{r}\fdel{v}{r}+ \sdel{v}{z}-\dfrac{v}{r^2}}}.
\end{align}
Now we will consider the boundary conditions and other simplifications to obtain a solution. First since we will consider the inertia-less where the fluid is moving relatively slow meaning that the angular velocity is close to zero which in turn implies $Re\to 0$ which when applied to equation (16) gives,
\begin{equation}
    \sdel{v}{r} + \dfrac{1}{r}\fdel{v}{r} - \frac{v}{r} + \sdel{v}{z} = 0,
\end{equation}
where $r\in\fbrac{0,1}, z\in\fbrac{0,\Gamma}$. Now considering the no-slip condition of the cylinder (meaning that the velocity on any stationary wall is 0) along with the rotating bottom the following boundary conditions for equation (17) are, 
\begin{align*}
    \text{Stationary sidewall at } r=1:  &\fang{u,v,w} = \fang{0,0,0}\nonumber\\
    \text{Stationary top at } z=\Gamma:  &\fang{u,v,w} = \fang{0,0,0}\nonumber\\
    \text{Rotating bottom at } z = 0:  &\fang{u,v,w} = \fang{0,r,0}\\
    \text{Axis } r=0:  &\fang{u,v,\fdel{w}{t}} = \fang{0,0,0}\nonumber
\end{align*}
or if we describe $v$ as a function of $r$ and $z$ we can write the conditions as,
\begin{align}
    v(0,z) &= v(1,z) = v(r,\Gamma) = 0\\
    v(r,0) &= r. 
\end{align}
Note that the boundary conditions for $\psi$ and $\eta$ as
\begin{align*}
    \psi(0,z,t) &= \psi(1,z,t) = \psi(r,0,t) = \psi(r,\Gamma, t) = 0 \\
    \eta(0,z,t) &= 0 \\
    \eta(1,z,t) &= -\dfrac{1}{r}\sdel{\psi}{r} \\
    \eta(r,0,t) &= -\dfrac{1}{r}\sdel{\psi}{z} \\
    \eta(r,\Gamma,t) &= -\dfrac{1}{r}\sdel{\psi}{z}.
\end{align*}
\section{Analytic Solution for Low Reynolds Number Flow}
We begin by noting we have non-homogeneous boundary conditions on the $z$ conditions. We can apply a shift to our boundary conditions to make the z boundary conditions homogeneous and the $r$ boundary conditions non-homogeneous. First we let $v(r,z) = f(r,z)+g(r,z)$ and assume that $f(r,z)$ is a bilinear function $f(r,z) = A +Br+Cz+Drz$ and rewrite the boundary conditions for $f(r,z)$ as  
\begin{align}
      f(r,\Gamma) &= 0 \nonumber\\
      f(r,0) &= r ,
\end{align}
such that 
\begin{align}
    g(r,\Gamma) = g(r,0) = 0.
\end{align}
When we apply the boundary conditions from equation (20) at $z=0$ and then at $z=\Gamma$ we have
\begin{align*}
    f(r,0) &= A+Br = r \implies A = 0, B = 1\\
    f(r,\Gamma) &=  r +C\Gamma+Dr\Gamma = 0\\
    &= r(1+D\Gamma)+C\Gamma =0 \implies C = 0, D = \dfrac{1}{\Gamma}.
\end{align*}
Since $f(r,z) = r(1+z/\Gamma)$ we can now solve for $g(r,z)$. To find the boundary conditions note that $g(r,z) = v(r,z)-f(r,z)$ meaning our boundary conditions for $g(r,z)$ are,
\begin{align}
    g(r,0) &= v(r,0) - f(r,0) = r-r = 0\\
    g(r,\Gamma) &= v(r,\Gamma) - f(r,\Gamma) = 0-0 = 0\\
    g(0,z) &= v(0,z) - f(0,z) = 0-0 = 0\\
    g(1,z) &= v(1,z)-f(1,z) = 0 - \fpar{1- \frac{z}{\Gamma}} = \frac{z}{\Gamma} -1  
\end{align}
From here we can use the method of separation of variables by assuming $g(r,z) = R(r)Z(z)$ and plugging it into equation (17) to get 
\begin{align}
    \dfrac{R''Z}{RZ}+\dfrac{1}{r}\dfrac{R'Z}{RZ} - \dfrac{RZ}{r^2} &=- \dfrac{RZ''}{Z}\nonumber\\
    \dfrac{R''}{R}+\dfrac{1}{r}\dfrac{R'}{R} - \dfrac{R}{r^2} &=\lambda = - \dfrac{Z''}{Z}
\end{align}
Here we will be looking at three cases for the eigenvalue $\lambda$.
\subsection{Applying $z$ Spacial Boundary Conditions}
\subsubsection*{CASE I: $\lambda = 0$}
Consider the $z$ ODE,
\begin{equation*}
    Z'' = 0 \implies Z = Az +B.
\end{equation*}
When applying the boundary conditions $Z(0)=Z(\Gamma) = 0 $ we have that, 
\begin{align*}
    Z(0) &= 0 = A(0) +B \implies B = 0\\    
    Z(\Gamma) &= 0 = A\Gamma \implies A = 0,
\end{align*}
hence $\lambda = 0$ only yields trivial solutions and will not be considered.

\subsubsection*{CASE II: $\lambda = -\omega^2<0$}
Considering the $z$ ODE again we have
\begin{align*}
    Z'' = \omega^2Z \implies Z = Ae^{\omega Z}+Be^{-\omega Z}
\end{align*}
Applying the same boundary conditions as in case I we have,
\begin{align*}
    Z(0) &= 0 = A+B \implies A=-B\\
    Z(\Gamma) &= 0 = A\fpar{e^{\omega Z}-e^{-\omega Z}} \implies A = 0.
\end{align*}
Since $\lambda = -\omega^2<0$ only yields trivial solutions it will not be considered. 

\subsubsection*{CASE III: $\lambda = \omega^2>0$}
Again consider the $z$ ODE, 
\begin{align*}
    Z''=-\omega^2Z \implies Z = A\cos\fpar{\omega z } + B\sin\fpar{\omega z }.
\end{align*}
Here when we apply the $z$ spacial boundary conditions we get,
\begin{align*}
    Z(0) &= 0 =  A\cos\fpar{0} + B\sin\fpar{0} = A \implies A = 0\\
    Z(\Gamma) &= 0 = B\sin\fpar{\omega \Gamma } \implies 0 = B\sin\fpar{\omega \Gamma } \implies \omega\Gamma = n\pi \implies \omega = \frac{n\pi}{\Gamma}, n\in\N.
\end{align*}
Here we see that there is a non-trival contribution for the $z$ ODE meaning,
\begin{equation}
    g(r,z) =\sum^\infty_{n=1} B_n\ R(r)\sin\fpar{\frac{n\pi z}{\Gamma}}.
\end{equation}
\subsection{Applying the $r$ Boundary Conditions}
Now we will consider the $r$ ODE described in equation (26) we can rewrite it using the definition $\lambda = \omega^2 = n^2\pi^2/\Gamma^2$ as,
\begin{equation}
    r^2R''+rR' -\fbrac{1 - \frac{n^2\pi^2}{\Gamma^2}r^2}R = 0.
\end{equation}
This is a known ODE known as the modified Bessel's equation which has the solution of 
\begin{equation*}
    R_n (r) = A\ I_1\fpar{\frac{n\pi}{\Gamma}r}+B\ K_1\fpar{\frac{n\pi}{\Gamma}r}, 
\end{equation*}
where $I_1(\frac{n\pi}{\Gamma}r)$ and $k_1(\frac{n\pi}{\Gamma}r)$ are the modified Bessel's functions of the first and second kind respectively of order one. Here we will apply the condition at $r=0$ which gives,
\begin{equation*}
    R_n(0) = 0 = A\ I_1(0) + B\ K_1(0).
\end{equation*}
Since we need bounded solutions for our system $K_1(\frac{n\pi}{\Gamma}r)$ is ignored since as $r\to0$, $K_1(r)\to\infty$ so we have now \begin{equation}
    g(r,z) = \sum_{n=1}^\infty C_n\ I_1\fpar{\frac{n\pi}{\Gamma}r}\sin\fpar{\frac{n\pi z}{\Gamma}}, C_n = A\cdot B_n.
\end{equation}  
Now we need to solve for $C_n$. We can do this by applying the condition at $g(1,z)$. Thus we have,
\begin{equation}
    g(1,z)= \frac{z}{\Gamma}-1 = \sum_{n=1}^\infty C_n\ I_1\fpar{\frac{n\pi}{\Gamma}}\sin\fpar{\frac{n\pi z}{\Gamma}}.
\end{equation}
We would like to apply orthogonality to this so we will multiply both sides of equation (30) by $\sin\fpar{m\pi z/\Gamma}$ to get, 
\begin{align}
    \fpar{\frac{z}{\Gamma} -1}\sin\fpar{\frac{m\pi z}{\Gamma}} &= \sum_{n=1}^\infty C_n\ I_1\fpar{\frac{n\pi}{\Gamma}}\sin\fpar{\frac{n\pi z}{\Gamma}}\sin\fpar{\frac{m\pi z}{\Gamma}}\nonumber\\
    \int_{0}^{\Gamma}{\fpar{\frac{z}{\Gamma} -1}\sin\fpar{\frac{m\pi z}{\Gamma}}}dz &= \sum_{n=1}^\infty C_n\ I_1\fpar{\frac{n\pi}{\Gamma}} \int_{0}^{\Gamma}{\sin\fpar{\frac{n\pi z}{\Gamma}}\sin\fpar{\frac{m\pi z}{\Gamma}}}dz.
\end{align}
Using half wave orthogonality,
\begin{equation}
    \int_{x_0}^{x_0+\lambda_1}\sin\fpar{\frac{m\pi z}{\Gamma}}\sin\fpar{\frac{n\pi z}{\Gamma}}dz = \frac{\lambda_1}{4}\delta_{mn},
\end{equation}
and knowing $\lambda_1 = 2\Gamma$ we have 
\begin{equation}
    \int_{0}^{2\Gamma}\sin\fpar{\frac{m\pi z}{\Gamma}}\sin\fpar{\frac{n\pi z}{\Gamma}}dz = \frac{\Gamma}{2}\delta_{mn},
\end{equation}
for the integral on the right hand side of equation (32) leaving us with,
\begin{align*}
        \int_{0}^{\Gamma}{\fpar{\frac{z}{\Gamma} -1}\sin\fpar{\frac{m\pi z}{\Gamma}}}dz &= \sum_{n=1}^\infty C_n\ I_1\fpar{\frac{n\pi}{\Gamma}} \frac{\Gamma}{2}\delta_{mn}\\
        &=C_m I_1\fpar{\frac{m\pi}{\Gamma}}\frac{\Gamma}{2}.
\end{align*}
Now looking at the left hand side of equation (31) we can use Mathematica to solve this integral for us using the following code
\begin{verbatim}
    Integrate[(z/gamma - 1) Sin[(m Pi z)/gamma], {z, 0, gamma}],
\end{verbatim}
which gives us that
\begin{equation*}
     \int_{0}^{\Gamma}{\fpar{\frac{z}{\Gamma} -1}\sin\fpar{\frac{m\pi z}{\Gamma}}}dz = -\frac{\Gamma}{m\pi}.
\end{equation*}
Thus equation (32) can be written as 
\begin{align*}
    -\frac{\Gamma}{m\pi} &= C_m I_1\fpar{\frac{m\pi}{\Gamma}}\frac{\Gamma}{2}\\
    C_m &= -\frac{2}{m\pi I_1\fpar{\frac{m\pi}{\Gamma}}}.
\end{align*}
Thus we have the solution for $g(r,z)$ being,
\begin{equation}
    g(r,z) =  - \dfrac{2}{\pi}\sum_{n=1}^\infty\fbrac{\dfrac{I_1\fpar{{n\pi r}/{\Gamma}}}{nI_1\fpar{{n\pi }/\Gamma}}\sin\fpar{\frac{n\pi z}{\Gamma}}}.
\end{equation}
\subsection{The Solution}
Using the principle of superposition we know that since $f(r,z)$ and $g(r,z)$ are both solutions to equation (17) the sum of the two is also a valid solution. Thus the solution to equation (17) given the boundary conditions described by (18) and (19) is 
%ANSWER%
\begin{equation}\label{eq:Exact_SS_Soln}
    v(r,z) = r\fpar{1-\dfrac{z}{\Gamma}} - \dfrac{2}{\pi}\sum_{n=1}^\infty\fbrac{\dfrac{I_1\fpar{{n\pi r}/{\Gamma}}}{nI_1\fpar{{n\pi }/\Gamma}}\sin\fpar{\frac{n\pi z}{\Gamma}}}.
\end{equation}

%\subsection{Plots of Varying $\Gamma$}
\begin{figure}[H]
\minipage{0.32\textwidth}
  \includegraphics[width=\linewidth]{Figures/MAT 462, ANALYTIC Gamma = 0 point 5.jpg}
  \caption{$v(r,z)$ with $\Gamma = 0.5 $}\label{fig:Gamma0.5}
\endminipage\hfill
\minipage{0.32\textwidth}
  \includegraphics[width=\linewidth]{Figures/MAT 462, ANALYTIC Gamma = 1.jpg}
  \caption{$v(r,z)$ with $\Gamma = 0.5 $}\label{fig:Gamma1}
\endminipage\hfill
\minipage{0.32\textwidth}%
  \includegraphics[width=\linewidth]{Figures/MAT 462, ANALYTIC Gamma = 2 POINT 5.jpg}
  \caption{$v(r,z)$ with $\Gamma = 2.5 $}\label{fig:Gamma2.5}
\endminipage
\end{figure}

From both the figures and from the the boundary conditions we can note that there is a singularity at $(r,z) = (1,0)$ since in the boundary conditions we define that point to be 0 with the condition $v(1,z) = 0$ and to be $r$ with the condition on $v(r,0)$. 
\subsection{Additional notes}
We can use an $l_2$ norm as a measure of stability for our numerical solution which can be given as, 
\begin{equation}\label{l2norm}
    ||\mathbf{l_{2}}|| = \sqrt{\dfrac{1}{(n_r+1)(n_z+1)} \sum_{i = 1}^{n_r}\sum_{j = 1}^{n_z}\fpar{v_{i,j}}^2},
\end{equation}
for a discrete array where $n_r,n_z\in\N$. Here we can see that as we add more terms to the series in equation \eqref{eq:Exact_SS_Soln} the $l_2$ norm converges in the following figures,


\begin{figure}[H]
\minipage{0.32\textwidth}
  \includegraphics[width=\linewidth]{Figures/MAT 462 PROJECT PART I G = 0_5 L2 NORM.png}
  \caption{$l_2$ norm against $\# \text{terms} $ with $\Gamma = 0.5 $}\label{fig:Gamma0.5l2}
\endminipage\hfill
\minipage{0.32\textwidth}
  \includegraphics[width=\linewidth]{Figures/MAT 462 PROJECT PART I G = 1 L2 NORM.png}
  \caption{$l_2$ norm against $\# \text{terms} $ with $\Gamma = 0.5 $}\label{fig:Gamma1l2}
\endminipage\hfill
\minipage{0.32\textwidth}%
  \includegraphics[width=\linewidth]{Figures/MAT 462 PROJECT PART I G = 2_5 L2 NORM.png}
  \caption{$l_2$ norm against $\# \text{terms} $ with $\Gamma = 2.5 $}\label{fig:Gamma2.5l2}
\endminipage
\end{figure}
Here we see that for each value of $\Gamma$ the $l_2$ norm converges as we increase $n_r$ and $n_z$.  

\section{Numerical Analysis for Low Reynolds Number Flow}
\subsection{Steady State Calculations using Method of Finite Differences}
Earlier we defined the partial differential equation to describe the flow of fluid in our cylindrical container 
\begin{equation}\label{eq:LOW RE TIME INDEP}
    \sdel{v}{r} + \dfrac{1}{r}\fdel{v}{r} - \frac{v}{r} + \sdel{v}{z} = 0,
\end{equation}
with boundary conditions, 
\begin{align}
    v(0,z) &= v(1,z) = v(r,\Gamma) = 0\\
    v(r,0) &= r. 
\end{align}
We made note that in our analytic solution there was a singularity at the point $(r,z) = (1,0)$ since our boundary conditions describe $v(1,0)$ being both 0 and 1. Since in nature we cannot have singularities we will apply numerical methods to get a more realistic solution. For this we will use the method of finite differences.

\subsection{Brief Overview of the Finite Difference Method}
Recall that for a smooth and continuous function $f(x)$ we have that the Taylor series approximation is
\begin{equation}\label{eq:TAYLOR}
    f(x+\delta_x) = f(x) +\delta_xf'(x) + \dfrac{\delta^2_x}{2!}f''(x) + \cdots + \dfrac{\delta^n_x}{n!}f^{(n)}(x) + \cdots.
\end{equation}
We will truncate the expansion after the second derivative to get the first order derivative's approximation which leaves us with,
\begin{equation}\label{eq:plus dx}
    f(x+\delta_x) = f(x) +\delta_xf'(x) + \dfrac{\delta^2_x}{2!}f''(x) + \mathcal{O}(\delta_x^3)    
\end{equation}
\begin{equation}\label{eq:minus dx}
    f(x-\delta_x) = f(x) -\delta_xf'(x) + \dfrac{\delta^2_x}{2!}f''(x) - \mathcal{O}(\delta_x^3).    
\end{equation}
Subtracting equation \eqref{eq:minus dx} from \eqref{eq:plus dx} we have 
\begin{align}
    f(x+\delta_x) - f(x-\delta_x) &= 2\delta_xf'(x) + \mathcal{O}(\delta_x^2) \nonumber \\
    \implies f'(x) &= \dfrac{f(x+\delta_x)-f(x-\delta_x)}{2\delta_x}+\mathcal{O}(\delta_x^2).
\end{align}
Next we'll expand equations \eqref{eq:TAYLOR} to the third order derivative to approximate the second order derivative as
\begin{equation}\label{eq:plus dx2}
    f(x+\delta_x) = f(x) +\delta_xf'(x) + \dfrac{\delta^2_x}{2!}f''(x) + \dfrac{\delta^3_x}{3!}f'''(x)+ \mathcal{O}(\delta_x^4) 
\end{equation}
\begin{equation}\label{eq:minus dx2}
    f(x-\delta_x) = f(x) -\delta_xf'(x) + \dfrac{\delta^2_x}{2!}f''(x) - \dfrac{\delta^3_x}{3!}f'''(x)+ \mathcal{O}(\delta_x^4)
\end{equation}
and adding equations \eqref{eq:plus dx2} and \eqref{eq:minus dx2} gives,
\begin{align}
    f(x+\delta_x) + f(x-\delta_x) &= 2f(x)+ \delta_x^2f''(x) + \mathcal{O}(\delta_x^4) \nonumber \\
    \implies f''(x) &= \dfrac{f(x+\delta_x)-2f(x)+f(x-\delta_x)}{2\delta_x^2}+\mathcal{O}(\delta_x^2).
\end{align}
Next we will discretize the meridonal plane into a uniform grid by stating that $(r_i,z_j) = (i\delta_r, j\delta_z)$ for $i\in\left\{1,2,\dots n_r\right\}$ and $j\in\left\{1,2,\dots n_z\right\}$ where $\delta_r = 1/n_r$ and $\delta_z = {\Gamma}/{n_z}$.
We'll use the notation $v_{i,j}$ for the function value, $v(r_i,z_j)$. With this we can rewrite equation \eqref{eq:LOW RE TIME INDEP}
as
\begin{align}
    \dfrac{v_{i+1,j} -2v_{i,j} + v_{i-1,j}}{\delta_r^2} + \dfrac{v_{i+1,j}-v_{i-1,j}}{2i\delta_r^2} - \dfrac{v_{i,j}}{i^2\delta^2_r} +\dfrac{v_{i,j+1} - 2v_{i,j} + v_{i,j-1}}{\delta_z^2} &= 0\nonumber\\
    \dfrac{1}{\delta_r^2}\fbrac{\fpar{1-\dfrac{1}{2i}}v_{i-1,j} - \fpar{2+\dfrac{1}{i^2}}v_{i,j} +\fpar{1+\dfrac{1}{2i}}v_{i+1,j}} + \dfrac{1}{\delta_z^2}\fbrac{v_{i,j-1}-2v_{i,j}+v_{i,j+1}} &= 0\\
    a_{n-1}v_{i-1,j} - a_{nn}v_{i,j} + a_{n+1}v_{i+1,j} + v_{i,j-1}\dfrac{1}{\delta_z^2} - v_{i,j}b_{mm} + v_{i,j+1}\dfrac{1}{\delta_z^2} &= 0,
\end{align}
using our definitions above. From here since there is a discrete number of points we can represent our equation above as the matrix equation
\begin{equation}\label{eq:Main Matrix eq}
    A_{nn}V_{nm}+V_{nm}B_{mm} = F_{nm},
\end{equation}
where $A_{nn}$ is the tridiagonal, $n\times n$, $r$-difference matrix, with the primary diagonal has elements $a_{nn}$ for $i\in[1,n]$, sub-diagonal elements $a_{n-1}$ for $i\in[2,n]$, and super-diagonal elements $a_{n+1}$ for $i\in[1,n-1]$; $B_{mm}$ is the tridiagonal $m\times m$, $z$-difference matrix, with main diagonal elements $b_{mm}$, and with sub and super diagonal elements being $1/\delta_z^2$; $V_{nm}$ being our solution to equation \eqref{eq:LOW RE TIME INDEP} of size $n\times m$ with entries $v_{i,j}$; and $F_{nm}$ is the right hand size with elements $f_{1,j}$ being $-i \delta_r/\delta_z^2$ for $i\in[1,n]$ as to account for the boundary condition $v(r,0) = r$ and zero elsewhere. Solving matrix equation \eqref{eq:Main Matrix eq} directly is resource intensive so diagonalizing matrix $A_{nn}$ by using a similarity transformation is used to speed up the calculations. \newline
Using a similarity transform we get $Z_{nn}^{-1}A_{nn}Z_{nn} = E_{nn}$ where $E_{nn}$ is the diagonal matrix containing the eigenvalues of $A_{nn}$ and $Z_{nn}$ is the matrix of the corresponding eigenvectors with $Z_{nn}^{-1}$ being it's inverse matrix. When we Substitute $V_{nm} =  Z_{nn}U_{nm}$ for matrix $U_{nm}$ is to determined later into \eqref{eq:Main Matrix eq} we have,
\begin{align}
    A_{nn}Z_{nn}U_{nm}+Z_{nn}U_{nm}B_{mm} &= F_{nm}\nonumber\\
    Z_{nn}^{-1}A_{nn}Z_{nn}U_{nm}+Z_{nn}^{-1}Z_{nn}U_{nm}B_{mm} &= Z_{nn}^{-1}F_{nm}\nonumber\\
    E_{nn}U_{nm}+U_{nm}B_{mm} &= Z_{nn}^{-1}F_{nm}\nonumber.
\end{align}
Here to take advantage of the symmetries of $E_{nn}$ and $B_{mm}$, we take the transpose of the matrix equation above resulting in 
\begin{equation}\label{eq:Matrix eq 2}
    B_{mm}U^T_{nm} + U^T_{nm}E_{nn} = Z_{nn}^{-1}F_{nm} = H_{mn}.
\end{equation}
From here we define a vector $\mathbf{u_i}$ being the rows of $U_{nm}$, $\mathbf{h_i}$ being the columns of $H_{mn}$, along with $e_i$ being the eigenvalues of $A_{nn}$, equation \eqref{eq:Matrix eq 2} can be written as 
\begin{equation}
    \fpar{B_{mm}+e_iI_{mm}}\mathbf{u_i} = \mathbf{h_i}, i\in[i,n].
\end{equation}
This equation can be solved faster than \eqref{eq:Main Matrix eq} and upon solving these $n$ equations for $U_{nm}$ we get the numerical answer since $V_{nm} = Z_{nn}U_{nm}$.

\subsection{Numerical Solutions}
Here you can see when programming the method described above in Python 3 we have the following diagrams,
\begin{figure}[H]
\minipage{0.32\textwidth}
  \includegraphics[width=\linewidth]{Figures/MAT 462 PROJECT PART II G = 0_5.png}
  \caption{Numerical solution to $v(r,z)$ with $\Gamma = 0.5 $}\label{fig:Gamma0.5}
\endminipage\hfill
\minipage{0.32\textwidth}
  \includegraphics[width=\linewidth]{Figures/MAT 462 PROJECT PART II G = 1.png}
  \caption{Numerical solution to $v(r,z)$ with $\Gamma = 1 $}\label{fig:Gamma1}
\endminipage\hfill
\minipage{0.32\textwidth}%
  \includegraphics[width=\linewidth]{Figures/MAT 462 PROJECT PART II G = 2_5.png}
  \caption{Numerical solution to  $v(r,z)$ with $\Gamma = 2.5 $}\label{fig:Gamma2.5}
\endminipage
\end{figure}
From here we note the accuracy of our numerical solution by taking the $l_2$ norm with the exact solution of $v(r,z)$. The relative error is defined as
\begin{equation}\label{eq:Rel_Error}
       \text{relative error} = \dfrac{l_{2}\fpar{v_{\text{exact}}} - l_{2}\fpar{ v_{\text{numerical}}}}{l_{2}\fpar{v_{\text{exact}}}}.
\end{equation}
First the $l_2$ norm was found to be for varying gammas,
\begin{table}[H]
\centering
\begin{tabular}{|c|c|}
        \hline
        $\Gamma$ & $l_2$ norm\\
        \hline
        0.5 & 0.2448185051721005  \\
        \hline
        1.0 & 0.18392053590295437 \\
        \hline
        2.5 &  0.11755681692232807  \\
        \hline
        \end{tabular}
\caption{\label{tab:l2norm}Values for $l_2$ norm for varying values of $\Gamma$.}
\end{table}
with the relative error being calculated as,
\begin{table}[H]
\centering
\begin{tabular}{|c|c|}
        \hline
        $\Gamma$ & relative error \\
        \hline
        0.5 & 0.011543499399148123 \\
        \hline
        1.0 & 0.020582494476482086 \\
        \hline
        2.5 &  0.024652470224409334   \\
        \hline
        \end{tabular}
\caption{\label{tab:RelErrorPartII}Relative for varying values of $\Gamma$.}
\end{table}

We can take note that if we take the limit as $n_r$ and $n_z$ approach infinity of equation \eqref{l2norm} and knowing that summations in equation \eqref{l2norm} converge as $n_r,n_z \to \infty$ to some value $v^2_{n_r,n_z}$ we have,
\begin{align*}
    \lim_{\substack{n_r \to\infty \\ n_z \to \infty}}\fbrac{\sqrt{\dfrac{1}{(n_r+1)(n_z+1)} \sum_{i = 1}^{n_r}\sum_{j = 1}^{n_z}\fpar{v_{i,j}}^2}} &=    \lim_{\substack{n_r \to\infty \\ n_z \to \infty}}\fbrac{\sqrt{\dfrac{1}{(n_r+1)(n_z+1)}}\sqrt{ \sum_{i = 1}^{n_r}\sum_{j = 1}^{n_z}\fpar{v_{i,j}}^2}}\\
    &= v_{n_r,n_z}\lim_{\substack{n_r \to\infty \\ n_z \to \infty}}{\sqrt{\dfrac{1}{(n_r+1)(n_z+1)}}}\\
    &= v_{n_r,n_z}\lim_{\substack{n_r \to\infty \\ n_z \to \infty}} \dfrac{1}{\sqrt{n_r+1}}\dfrac{1}{\sqrt{n_z+1}}\\
    &= 0.
\end{align*}

These results show two things. First is that the error of our numerical solution is at or near the order of $10^{3}$. This high amount of error led to further investigation into why and where the error propagated to this degree. To visualize where the error was the highest a plot of of the relative error was made per value of $\Gamma$ which can be seen below, and second that as we make our grid finer, the $l_2$ norm vanishes.
\begin{figure}[H]
\minipage{0.32\textwidth}
  \includegraphics[width=\linewidth]{Figures/MAT 462 PROJECT PART II G = 0_5 ERROR.png}
  \caption{Numerical solution to $v(r,z)$ with $\Gamma = 0.5 $}\label{fig:Gamma0.5err}
\endminipage\hfill
\minipage{0.32\textwidth}
  \includegraphics[width=\linewidth]{Figures/MAT 462 PROJECT PART II G = 1 ERROR.png}
  \caption{Numerical solution to $v(r,z)$ with $\Gamma = 1 $}\label{fig:Gamma1err}
\endminipage\hfill
\minipage{0.32\textwidth}%
  \includegraphics[width=\linewidth]{Figures/MAT 462 PROJECT PART II G = 2_5 ERROR.png}
  \caption{Numerical solution to  $v(r,z)$ with $\Gamma = 2.5 $}\label{fig:Gamma2.5err}
\endminipage
\end{figure}
Here we see that a majority of the error is at or near the singularity so when summing over all the whole plane most of the error accumulation was near $(r,z)=(1,0)$.
\subsection{Transient Solution for Small Reynolds Numbers}  
To model the time evolution of our system, we used the method of lines. The method of lines (MOL for short), is the method of finding solutions to partial differential equations where we discretize all but one dimension (in our case the time dimension) and make a system of ODE's in which we can numerically integrate. We start by discretizing the PDE
\begin{equation}\label{eq:Time evo Linear}
    \fdel{v}{t} = \dfrac{1}{Re}\fbrac{\sdel{v}{r} + \dfrac{1}{r}\fdel{v}{r} - \dfrac{v}{r^2} + \sdel{v}{z}},
\end{equation}
giving us the equation for the interior gird points, 
\begin{align}
    \fdel{v_{i,j}}{t} &= \dfrac{1}{Re} \fbrac{\dfrac{v_{i+1,j} -2v_{i,j} + v_{i-1,j}}{\delta_r^2} + \dfrac{v_{i+1,j}-v_{i-1,j}}{2i\delta_r^2} - \dfrac{v_{i,j}}{i^2\delta^2_r} +\dfrac{v_{i,j+1} - 2v_{i,j} + v_{i,j-1}}{\delta_z^2}}\nonumber\\
    &= \text{RHS}(v_{i,j}). 
\end{align}
Here we make a function called RHS($v_{i,j}$) which will take in the array containing the exterior grid points and the interior grid points and will return another array. This here creates the system of linear ODE's we will numerically integrate. To do this Heun's method was used for the numerical integration. How Heun's method works is that it is a two step predictor-corrector method. First we predict where the next point of the function (time step $k+1$) by doing the following at time step $k$
\begin{equation}\label{eq:predict}
    v^p_{i,j} = v^k_{i,j} + \delta_t \cdot \text{RHS}\fpar{v^k_{i,j}}.
\end{equation}
From here we use the result from equation \eqref{eq:predict} to correct our prediction by doing a similar process and averaging the previous step from the predicted step as seen below,  

\begin{equation}\label{eq:Correct}
    v^{k+1}_{i,j} = v^k_{i,j} + \dfrac{\delta_t}{2} \cdot\fpar{\text{RHS}\fpar{v^p_{i,j}}+ \text{RHS}\fpar{v^k_{i,j}}}.
\end{equation}
A video of the time evolution of the system can be found  \href{https://youtu.be/zd8_BHrUcXo}{HERE}.
\newline
Upon testing it was noted that numerical stability happens on time steps
\begin{equation*}
    \mathcal{O}(\delta_t) = \max\fset{\mathcal{O}\fpar{\dfrac{1}{10\cdot Re \ \delta_r^2}},\mathcal{O}\fpar{\dfrac{1}{10\cdot Re \ \delta_z^2}}}.
\end{equation*}
Calculating the relative error between the time evolution code and the steady state code was done by using the follwoing,
\begin{equation*}
    \text{relative error} = \dfrac{l_{2}\fpar{v_{\text{time evo}}} - l_{2}\fpar{ v_{\text{steady state}}}}{l_{2}\fpar{v_{\text{time evo}}}}.
\end{equation*}
This result showed that the $l_2$ norm of the relative difference from the steady state solution and the time evolution code was $0.04820476770339076$. This is a high error which means that the code used for time evolution does not have a fine enough time-step to to better approximate the value given by the steady state code.
Another result to be noted is that when running different Reynolds numbers (specifically $Re = 10$ and $Re = 100$) the time to reach steady state took longer than when using $Re = 1$.



\section{Numerical Analysis of Swirling Flow}

\section{Appendix}
\subsection{Documentation of Code}
All code documentation can be found at the github link   \href{https://github.com/Thesaxman1126/MAT-462-Project}{HERE}.
\end{document}
